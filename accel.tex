%Tudo que começa com '%' é comentário e é ignorado pelo compilador

%Gerando arquivo em latex:
%latex arquivo.tex (em dvi)
%pdflatex arquivo (em pdf)
%dvipdfm arquivo
%s2pdf arquivo

% Alguns modos de usar o latex:
% Windows – Miktex com Led
% Linux – texlive com kile

\documentclass[12pt]{report} %aqui fala o tipo de documento e o tamanho da fonte. Opções: tamanho do texto (10pt, 12pt, 14pt), formato do papel (a4paper, a5paper, b5paper, letterpaper, legalpaper, executivepaper), o número de colunas (onecolumn, twocolumn), entre outras opções.
%Por exemplo, [12pt,a4,twocolumn].
%classe: article, report, letter, book ou slides. Instalar abnt para quem está pensando no tf
\usepackage[brazilian]{babel} %hifenização em português do brasil
\usepackage[T1]{fontenc} % caracteres com acentos são considerados um bloco só
\usepackage[utf8]{inputenc} % Corrigie os acentos em Português
\usepackage{ae} %arruma a fonte quando usa o pacote fontenc
%\usepackage[pdftex]{graphicx}%Para inserir figuras
\usepackage{vhistory}

\begin{document}
\title{ICTP - Aceleradores 2015}
\author{Vitor Vasconcelos Araújo Silva\\ vitors@cdtn.br}
\date{\today}

\maketitle %cria o título

%\def \negritovi {\textbf} %Criando comandos

\tableofcontents %índice
\pagebreak % Quebra de página
%\listoffigures %indice de figuras
%\listoftables %indice de tabelas
\pagebreak % Quebra de página

\begin{abstract}
  Este documento armazena - sem maior organização - as anotações feitas
  no curso sobre GPUs no ICTP em 2015.
\end{abstract}

\begin{versionhistory}
	\vhEntry{0.0}{30/11/2018}{Vitor}{Documento criado}
%	\vhEntry{1.1}{23.01.04}{DP|JPW}{correction}
%    \vhEntry{1.2}{03.02.04}{DP|JPW}{revised after review}
\end{versionhistory}

%\include{seção}

\end{document}
